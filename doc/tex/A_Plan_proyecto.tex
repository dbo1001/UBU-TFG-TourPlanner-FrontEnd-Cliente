\apendice{Plan de Proyecto Software}

\section{Introducción}

En este apartado se tratarán varios temas importantes a la hora de realizar un trabajo. 

Uno de ellos será la planificación temporal, a través de la cual se sabrá la asignación de tiempos que realizar durante todo el trascurso del proyecto.

Todo el trabajo se desarrolla en torno a metodologías de trabajo como SCRUM, con la cual se puede conseguir una identificación mas sencilla de la asignación de tareas y tiempos y la inversión de tiempo en proporción al total de horas que llevará el proyecto.

Otro es el estudio de la viabilidad, donde se tratarán dos puntos esenciales, la viabilidad económica y la viabilidad legal. Por la parte económica se tratará de simular las consecuencias que conlleva manejar unas herramientas u otras a la hora de invertir dinero en el proyecto o de que otras personas vayan a invertirlo cuando se trate venderlo.

La viabilidad legal se centrará un poco más en el estudio de los derechos de copyright, así como el trato que se haga de los datos. Por último también se tendrá en cuenta la seguridad en cuando a protección de los datos y las capas de seguridad del sistema que tratamos.

\section{Planificación temporal}

El objetivo inicial al planificar un proyecto es definir lo más claro posible el marco de tiempo y de recursos que se llevarán a cabo durante el desarrollo del proyecto. Como ya mencionábamos antes, se basa en la metodología SCRUM.

Dado que este proyecto ha sufrido bastantes cambios sobre la planificación y desarrollo del mismo desde que se comenzó hasta la actualidad, haremos una comparación entre lo que inicialmente se planeó y lo que finalmente resulta, a fin de poder ver todo lo que un trabajo como este, donde no se definen unos objetivos claros desde un inicio, o se definen objetivos sin planificar las consecuencias puede variar y evolucionar.

El proyecto comienza el día 5 de abril de 2019 y finaliza el día 13 de febrero de 2020, con una duración bruta de 10 meses. Ésta duración también se debe, como se ha mencionado antes, a los cambios que ha ido sufriendo el desarrollo del proyecto. Igualmente, también hay que tener en cuenta que no se ha dedicado todo el tiempo a la realización del proyecto, ya que también era necesario aprobar las asignaturas restantes así como realizar prácticas laborales durante todo este período. También se terminó aplazando la entrega debido a la imposibilidad de presentar en la convocatoria de Septiembre por no cumplir los requisitos de créditos máximos presentables en la misma.

Por todo esto que se expone, la realización del proyecto ha tenido una evolución irregular, con meses más activos que otros debido a la complicación para realizar reuniones entre los tutores y los alumnos participantes del mismo.

Se pueden ver las partes que ha conllevado:

\begin{itemize}

\item SCRUM se basa en la división de las tareas y recursos en intervalos de tiempo, por lo que será como se desarrolle. 

\item Cada uno de los sprints conllevará unas tareas a realizar durante ese intervalo. Se trató de que las reuniones se realizaran cada 2 semanas, o una vez al mes.

\end{itemize}

El comienzo del proyecto se realizó con una reunión estándar con el tutor, que nos comentó el punto de partida y los objetivos iniciales que podríamos tener para comenzar el trabajo.

Cabe destacar que se explicará cada uno de los sprints partiendo siempre de una reunión inicial, y estableciendo las nuevas tareas que se realizarán durante el mismo.

\subsection{Sprint 1 (05/04/19 - 19/04/19)}

En esta primera reunión el tutor explica los objetivos iniciales y se establecen las siguientes tareas:

\begin{itemize}
\item Búsqueda de documentación sobre Android.
\item Desarrollo de objetivos iniciales de cara a la memoria.
\item Consulta inicial del proyecto base del que partíamos.
\end{itemize}

\subsection{Sprint 2 (19/04/19 - 03/05/19)}

Tras el primer sprint, pudimos comentar el conocimiento que cada uno tenía sobre la plataforma sobre la que trabajaríamos así como el lenguaje de programación. En mi caso, nunca había utilizado el programa de Android Studio ni programado para un entorno Android, aunque si que conocía el lenguaje de programación Java.

Se propuso lo siguiente:

\begin{itemize}
\item Comenzar a nutrirse de tutoriales y vídeos explicativos sobre Android.
\item Instalar la herramienta de Android Studio.
\item Pensar en la viabilidad de actualizar la versión en la que se encuentra el proyecto, así como las posibilidades de cambiar de herramienta de desarrollo, porque en ese momento el proyecto se había desarrollado en Eclipse, con un plugin para Android debido a la inexistencia de Android Studio en esa época.
\end{itemize}

\subsection{Sprint 3 (03/05/19 - 17/05/19)}

En este encuentro, se trata las posibilidades de lo comentado en la reunión anterior, concluyendo que el cambio de herramienta de desarrollo será algo positivo y relativamente fácil de hacer, pero sin tener claro las posibilidades sobre la actualización de versiones. 

Se trata de desarrollar lo siguiente para el siguiente sprint:

\begin{itemize}
\item En caso de plantear la actualización, buscar la versión actual más óptima hacia la que se debería hacer, para evitar versiones inestables y con bugs.
\item Instalar el proyecto inicial a través de maquinas virtuales y probar todas las funcionalidades que contiene, para conocer bien sobre lo que vamos a trabajar.
\item Seguir desarrollando apartados referentes de la memoria para ir quitando carga de trabajo al final.
\end{itemize}

\subsection{Sprint 4 (17/05/19 - 07/06/19)}

En esta reunión, tras tratar los temas anteriores, se concluye que la versión a la que actualizar debe ser Android 6.0 (API 23), viniendo desde android 4.4 (API 14). Supone un gran salto pero si queremos que la aplicación sea funcional con la tecnología de hoy en día habrá que usar una versión viable. Sobre este tema se plantea el hecho de la antigüedad del proyecto, teniendo en cuenta esto para el resto del desarrollo.

Se intenta realizar lo siguiente:

\begin{itemize}
\item Instalación del proyecto base sobre la nueva herramienta de desarrollo, para intentar que este quede ya funcional en Android Studio.
\item Documentación sobre cambios de versiones en Android hasta la versión sobre la que se desarrolla.
\end{itemize}

\subsection{Sprint 5 (07/06/19 - 05/07/19)}

Reunión cercana a los meses de verano, por lo que se intenta dejar objetivos más a largo plazo, aunque también comentando las dificultades que se van encontrando a la hora de instalar y hacer que funcione el proyecto sobre la nueva herramienta:

\begin{itemize}
\item Conseguir que definitivamente la aplicación alcance un estado de funcionamiento correcto sobre Android Studio.
\item Aplicar el cambio de versión sobre el proyecto y por consiguiente comenzar a solucionar todos los errores de base que se encuentren.
\end{itemize}

\subsection{Sprint 6 (05/07/19 - 03/09/19)}

En esta reunión, tras observar los problemas que supone hacer que funcione la aplicación sobre Android Studio y por consiguiente actualizarla, los objetivos de la anterior reunión siguen siendo los mismos con algunos añadidos.

El problema principal parece ser que la gran mayoría de librerías que se utilizaron en el proyecto inicial han dejado de tener soporte y han dejado de funcionar, por lo que todo ese código es ahora inservible y habrá que buscar una sustitución.

Se plantean los objetivos anteriores añadiendo:

\begin{itemize}
\item Desarrollo de la memoria, tratando de quitar carga a futuro.
\item Arreglo de errores con bibliotecas.
\item Documentación sobre nuevas APIs utilizables para sustituir todas las que ya están obsoletas.
\end{itemize}

\subsection{Sprint 7 (03/09/19 - 04/11/19)}

Una reunión tratando de ver los avances del verano, para después poner objetivos a largo plazo. En este momento ya se presentan más dificultades para trabajar en el proyecto y tener reuniones habituales por diversos motivos.

A partir de los objetivos que se tenían inicialmente, se intenta establecer unos más realistas teniendo en cuenta la situación:

\begin{itemize}
\item Continuar con todos los arreglos necesarios sobre la aplicación, ya que según se van solucionando algunas incompatibilidades, se van descubriendo nuevas.
\item Intentar continuar el desarrollo de la memoria.
\end{itemize}

\subsection{Sprint 8 (04/11/19 - 18/11/19)}

Tras el tiempo que ha pasado, se intenta tener un seguimiento sobre las tareas que habían sido asignadas. Por desgracia, estas tareas no varían demasiado, ya que supone un trabajo importante la documentación sobre los métodos que tienen que ser sustituidos para incluir otro nuevos, sobre los que también es necesario documentarse previamente, sobre un lenguaje de programación relativamente nuevo.

\subsection{Sprint 9 (18/11/19 - 02/12/19)}

Tras haber ido trabajando, en esta nueva reunión se presenta un problema que ha ido siendo recurrente durante el desarrollo del proyecto, ya que aunque se han solucionado muchos problemas, la conexión entre la parte del cliente y la del servidor no parece funcionar, lo cual genera bastante incertidumbre porque se desconoce si la causa puede venir desde el cliente o desde el servidor y este problema se va arrastrando.

Objetivos anteriores añadiendo:

\begin{itemize}
\item Tratar de arreglar el problema de la conexión, identificando el origen y tratándolo.
\end{itemize}

\subsection{Sprint 10 (02/12/19 - 16/12/19)}

Reunión sin demasiado contenido, ya que se sigue tratando de realizar las mismas tareas, por lo que los objetivos que ya se tenían, se trasladan a este sprint.

\subsection{Sprint 11 (16/12/19 - 13/01/20)}

Una de las ultimas reuniones que se producen, tratando de ver los últimos objetivos que se tienen de cara a finalizar el trabajo, así como viendo si es posible arreglar la conexión. 

Se detectan varios problemas que pueden llegar a ser la causa, aparte de la diferencia de versiones entre la aplicación y el servidor que, a la hora de comunicarse generan problemas.

\begin{itemize}
\item Tratándose de fechas cercanas a la entrega, desarrollar la memoria.
\item Solucionar conexión.
\item Depurar el código.
\item Cambios visuales propuestos sobre la aplicación para mejorar la accesibilidad de la misma.
\end{itemize}

\subsection{Sprint 12 (13/01/20 - 07/02/20)}

En esta reunión se establece los últimos objetivos, así como una fecha para entregar la memoria a los tutores para que puedan revisarla y proponer mejoras.

\begin{itemize}
\item Continuar depurando código.
\item Finalizar memoria.
\end{itemize}

\subsection{Sprint 13 (07/02/20 - Fin de proyecto)}

Para la última reunión se proponen cambios en la memoria y se realiza una visión general sobre el proyecto.

\begin{itemize}
\item Finalizar código.
\item Realizar mejoras en memoria.
\item Pancarta.
\end{itemize}


Ya se ha visto que el trascurso de este proyecto ha tenido épocas más activas que otras, debido a las complicaciones que se comentan. De todas formas se ha tratado de seguir la metodología de la mejor manera posible.

\section{Estudio de viabilidad}

Como ya hemos comentado antes, en este apartado se trata la viabilidad del proyecto, de manera económica y legal, ya que cabe la posibilidad de que un proyecto posea las cualidades para ser interesante de realizar, pero no ser viable económicamente o legalmente debido a protección de ciertos u otros factores.

Hay otro factor, que en el caso concreto de este proyecto adquiere mucha importancia, que es la viabilidad de realizar el proyecto a nivel de magnitud y conocimiento. Lo trataremos inicialmente:

\subsection{Viabilidad sobre la magnitud y conocimiento del proyecto}

Hay que tener en cuenta una serie de factores en este apartado:

\begin{itemize}

\item Conocimiento sobre el lenguaje que se va a utilizar.
\item Conocimiento sobre la herramienta que se va a utilizar.
\item Análisis sobre la viabilidad de desarrollo del proyecto en cuestión, estableciendo límites de tiempos y recursos.
	
	Con respecto a esta parte, es algo que a la hora de planificar un proyecto hay que tener muy en cuenta, porque aunque pueda resultar algo muy interesante, puede no ser viable debido a la complejidad del mismo.
	
	En nuestro caso, si se hubiera realizado de forma más minuciosa este apartado, probablemente se habría llegado a la conclusión de que es un proyecto de final de carrera muy poco viable, porque supone obtener conocimiento sobre herramientas y lenguajes nunca utilizados, analisis de un codigo que no es propio, desarrollado en versiones con mas de 7 años de antigüedad, tratar de actualizarlo sin conocer perfectamente todas las dificultades que se pueden encontrar, etc.
	
	Por todo lo mencionado se plantea este apartado, para que una persona que comience un proyecto, trate minuciosamente la información de la que dispone, y analice las ventajas e inconvenientes que supone, así como el tiempo del que dispondrá para su desarrollo.

\end{itemize}



\subsection{Viabilidad económica}

Dado que partimos de un proyecto anterior, es conveniente mencionarlo, ya que en él se realiza un análisis sobre la viabilidad económica bastante acertado y que contiene información perfectamente válida para el estudio de la viabilidad de nuestro proyecto.

\subsubsection{Análisis de costes}

Tal como se menciona en el proyecto anterior, tendremos en cuenta 4 tipos de costes:

\begin{itemize}
\item Coste de personal
\end{itemize}

La persona que se encargará de realizar el proyecto será un desarrollador de software, y como hemos comentado que el proyecto tiene una duración de unos 10 meses, con una media de 4 horas diarias trabajadas tendremos una estimación bastante decente, con épocas mas activas y otras menos. En este caso tomaremos de referencia coste/hora del proyecto anterior que suponía 12 euros/hora.

\subsection{Viabilidad legal}

Para la viabilidad legal del proyecto, nos tendremos que fijar en la utilización de los datos y las licencias que se utilicen para desarrollar el trabajo.

Tal como se menciona aquí \cite{terminos}, para poder utilizar todas las funcionalidades que se nos ofrecen debemos aceptar ciertas condiciones y hacer un uso responsable de las mismas, para no infringir causas legales.

Para ello, dentro del propio proyecto desarrollado en Android Studio contaremos con archivos que harán referencia a estas licencias, y que todo proyecto que se desarrolle con estas tecnologías deberá incluir como parte de su estructura.


