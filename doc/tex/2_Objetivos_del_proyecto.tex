\capitulo{2}{Objetivos del proyecto}

En este apartado explicaré los distintos objetivos que se van a tener en cuenta en este proyecto, diferenciando entre las posibles mejoras a realizar en la aplicación y los objetivos mas personales de aprendizaje y desarrollo como programador en base al trabajo realizado.

Objetivos de mejora:
\begin{itemize}
\item Actualización de la interfaz de proyecto desde un sistema Android obsoleto hacia la version mas actualizada posible del mismo, pudiendo implementar mejoras tanto visuales como funcionales
\item Actualización sobre la utilización de una galería de imágenes relacionadas con los puntos de interes, ya que la ya implementada era utilizando el servicio de Panoramio, y habría que valorar si esos servicios siguen estando disponibles o si es necesario valerse de otros servicios, como Flickr o Instagram.
\item Diferenciación entre una ruta y otra si se ha realizado en la misma ciudad, para que la aplicación calcule lugares diferentes basándose en los ya visitados.
\item Implementación de aplicaciones externas como Twitter o TripAdvisor para poder descubrir las opiniones o valoraciones de otros usuarios.
\item Diferenciación entre puntos de interés basándose en la actualización del algoritmo que pasa a utilizar horarios de apertura y cierre de los mismos, para que la aplicación no visualice ningún lugar que no pueda ser visitado por estos factores.
\item Estandarización de los métodos que se utilizan y de la interfaz que se utiliza en la aplicación.
\item Optimización de las clases y tipos que se utilizan, ya que desde la version de Android en la que se implementó originalmente la aplicación han quedado varios obsoletos y han aparecido otros nuevos mas efectivos.
\item Aplicar técnicas de refactorización en el código que ya hay desarrollado en la aplicación.
\end{itemize}

Objetivos personales:
\begin{itemize}
\item Aprender a utilizar el entorno y lenguaje de programación de Android, ya que es un sistema que puede resultar muy útil para la vida laboral de un programador.
\item Mejorar en el proceso de la gestión de tareas, con aplicaciones como GitHub.
\item Aprender a utilizar el entorno de desarrollo de documentación LaTeX, ayudándome de la herramienta Texmaker.
\item Aprender a trabajar en modo SCRUM, utilizando Issues y Sprints para la realización del proyecto.
\item Aprender a realizar trabajo en equipo, manteniendo comunicación con el compañero encargado de desarrollar el algoritmo del proyecto, para conseguir experiencia sobre un proyecto mas realista que si se realizara solo.
\item Conseguir mayor conocimiento sobre técnicas de refactorización y el software que se utiliza para detectar defectos de código.
\end{itemize}

