\capitulo{7}{Conclusiones y Líneas de trabajo futuras}
Se concluirá este proyecto indicando una serie de aspectos desde un punto de vista más personal que en el resto de apartados:

\section{Conclusiones}

Se va a intentar dividir este apartado entre las conclusiones más técnicas y las más personales.

\subsubsection{Conclusiones técnicas}

De cara a la experiencia para el usuario, con este proyecto se ha conseguido:

\begin{itemize}
\item Actualización de la aplicación, por lo que podrá ser instalada en los dispositivo actuales.
\item Actualización de la interfaz, para que resulte más familiar, pareciéndose más a cómo es la interfaz actual de Android.
\item Optimización, ya que se ha mejorado el código, eliminando partes que afectaban negativamente y reduciendo el numero de code smells.
\end{itemize}

\subsubsection{Conclusiones personales}

En esta sección, me gustaría mencionar lo complicado que puede resultar la reutilización de un proyecto en el que han participado varios autores. Esto se debe, entre otras cosas, a las diferencias en la manera de estructurar, pensar o desarrollar el código de la aplicación. Enfrentarse a un reto como este puede aportar mucha experiencia de cara al mundo laboral, ya que cuando se trata de desarrollar proyectos fuera del ámbito universitario será lo más parecido que nos encontremos.

De igual forma, también me ha servido para aprender sobre el mundo de la programación en Android, partiendo desde cero, lo cual supone una experiencia muy enriquecedora.

En ese sentido esta experiencia ha hecho que maduremos bastante como programadores y también ha aportado una visión diferente en la forma de programar, aunque inicialmente resulta bastante más dificultoso de lo que pudiera parecer.

Por último, me gustaría destacar el esfuerzo realizado para poder desarrollar el proyecto en conjunto con mi compañero, teniendo que trabajar en remoto y aportando información para la ayuda en un proyecto o en el otro.

\section{Líneas de trabajo futuras}

En este apartado se destacarán algunas ideas que se podrían desarrollar en un futuro partiendo con este proyecto como base. Se indicarán algunas propuestas del proyecto anterior \cite{tfm1}, ya que aunque no se han podido llevar a cabo para este proyecto, sería bastante interesante implementarlas.

\begin{itemize}
\item Implementación de la aplicación para iOS y Windows Phone, ampliando la integración de nuestra aplicación con los distintos sistemas operativos para móviles.
\item Creación de una página web que iguale la funcionalidad de nuestra aplicación.
\item Mejorar el buscador de puntos de interés añadiendo diversos filtros como pueden ser la calle, la valoración, la categoría, etc…
\item Inclusión de realidad aumentada en la aplicación, pudiendo así visualizar los puntos de interés cercanos a medida que se desplaza o gira el móvil (por ejemplo, en un giro de 360º se verían unos puntos u otros dependiendo de la posición en la que se encuentre el usuario).
\item Diferenciación entre una ruta y otra si se ha realizado en la misma ciudad, para que la aplicación calcule lugares diferentes basándose en los ya visitados.
\item Integrar la API de Microsoft Power BI \cite{pbi}, para poder visualizar un análisis con gráficas sobre los puntos de interés más visitados de una ciudad, etc.
\end{itemize}
