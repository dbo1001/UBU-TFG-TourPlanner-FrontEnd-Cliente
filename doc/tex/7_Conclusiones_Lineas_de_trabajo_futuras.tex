\capitulo{7}{Conclusiones y Líneas de trabajo futuras}

Todo proyecto debe incluir las conclusiones que se derivan de su desarrollo. Éstas pueden ser de diferente índole, dependiendo de la tipología del proyecto, pero normalmente van a estar presentes un conjunto de conclusiones relacionadas con los resultados del proyecto y un conjunto de conclusiones técnicas. 
Además, resulta muy útil realizar un informe crítico indicando cómo se puede mejorar el proyecto, o cómo se puede continuar trabajando en la línea del proyecto realizado. 

Podemos concluir este proyecto indicando una serie de aspectos.

\subsection{Código con múltiples autores}

En esta sección, me gustaría mencionar lo complicado que puede resultar la reutilización de un proyecto en el que han participado varios autores. Esto se debe, entre otras cosas a las diferencias en la manera de estructurar, pensar o desarrollar el código de la aplicación. Enfrentarse a un reto como este puede aportar mucha experiencia de cara al mundo laboral, ya que cuando se trata de desarrollar proyectos fuera del ámbito universitario será lo más parecido que nos encontremos.

En ese sentido esta experiencia ha hecho que maduremos bastante como programadores y también ha aportado una visión diferente en la forma de programar, aunque inicialmente resulta bastante más dificultoso de lo que pudiera parecer.
