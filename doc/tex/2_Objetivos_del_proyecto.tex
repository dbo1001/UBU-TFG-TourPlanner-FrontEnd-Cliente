\capitulo{2}{Objetivos del proyecto}

En este apartado se explicarán los distintos objetivos que se van a tener en cuenta en este proyecto, diferenciando entre las posibles mejoras a realizar en la aplicación y los objetivos mas personales de aprendizaje y desarrollo como programador en base al trabajo realizado.

\textbf{Objetivos de mejora:}
\begin{itemize}
\item Actualización de la interfaz del proyecto desde un sistema Android obsoleto hacia la versión más actualizada posible del mismo, pudiendo implementar mejoras tanto visuales como funcionales.
\item Actualización sobre la utilización de una galería de imágenes relacionadas con los puntos de interés, ya que la ya implementada utilizaba el servicio de Panoramio, y habría que valorar si esos servicios siguen estando disponibles o si es necesario valerse de otros, como Flickr o Instagram.
\item Búsqueda de librerías nuevas para sustituir las que se encuentran obsoletas.
\item Implementación de estas nuevas librerías, ajustando el código necesario.
\item Estandarización de los métodos y la interfaz desarrollados en la aplicación.
\item Optimización de las clases y tipos que se utilizan, ya que desde la versión de Android en la que se implementó originalmente la aplicación, han quedado varios obsoletos y han aparecido otros nuevos más efectivos.
\item Aplicar técnicas de refactorización en el código de la aplicación.
\item Implementación de aplicaciones externas como Twitter o TripAdvisor para poder descubrir las opiniones o valoraciones de otros usuarios.
\end{itemize}

\textbf{Objetivos personales:}
\begin{itemize}
\item Aprender a utilizar el entorno y lenguaje de programación de Android, ya que es un sistema que puede resultar muy útil para la vida laboral de un programador.
\item Mejorar en el proceso de la gestión de tareas, con aplicaciones como GitHub.
\item Aprender a utilizar el entorno de desarrollo de documentación LaTeX, ayudándome de la herramienta Texmaker.
\item Mejorar en el trabajo con metodología SCRUM, utilizando Issues y Sprints para la realización del proyecto.
\item Aprender a realizar trabajo en equipo, manteniendo comunicación con el compañero encargado de desarrollar el algoritmo del proyecto \cite{tfg2}, para conseguir experiencia sobre un proyecto más realista que si se realizara solo.
\item Conseguir mayor conocimiento sobre técnicas de refactorización y el software que se utiliza para detectar defectos de código.
\end{itemize}

