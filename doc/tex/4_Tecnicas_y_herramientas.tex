\capitulo{4}{Técnicas y herramientas}


En este apartado se destacarán las principales técnicas y herramientas utilizadas para el proyecto. Cabe destacar que habrá herramientas también utilizadas que no serán explicadas, bien por pertenecer a la parte del servidor o por ser herramientas que fueron utilizadas en la anterior versión de este trabajo \cite{tfm1}, pero que para esta versión sus funcionalidades no se han modificado y por tanto no se ha aportado nada adicional a lo que ya se implementó.

\subsection{Técnicas}

Las técnicas utilizadas durante el desarrollo de un proyecto pueden ayudar a comprender mejor la metodología que se ha llevado a cabo, así como la robustez que tiene el código. También nos permiten observar ligeramente el rumbo que ha tomado el proyecto, ya que puede ser un proyecto mas enfocado a la implementación de métodos desde cero, o para complementar un proyecto con un ciclo de vida mas largo. Por otro lado, también se puede estar buscando el mantenimiento, optimización y actualización del código que ya esta desarrollado en un proyecto, como es nuestro caso.

\subsubsection{Refactorización}

La refactorización \cite{refac1} \cite{refac2} es una técnica utilizada en programación para realizar modificaciones sobre el código de un programa en cuanto a la estructura interna, pero sin que el comportamiento externo se vea afectado. Son cambios que se pueden apreciar estructuralmente y que a simple vista podrían no tener importancia, pero a medida que un programa avanza, se actualiza o se dejan de usar ciertas técnicas para pasar a usar otras, haber hecho esta refactorización cobra mas importancia. 

Si nos paramos a pensar, el termino de refactorización tiene el origen en las matemáticas, ya que la "factorización" en las matemáticas lo que puede hacer es convertir un polinomio compuesto por una variable y un numero en 2 polinomio mas simples y con una estructura mas clara de ver. El funcionamiento externo es en sí el mismo, pero se consigue una limpieza o una robustez estructural. Esto es lo que se busca al hacerlo en la programación.

La refactorización abarca desde cambiar el nombre de una variable para que vaya mas acorde con la función que llevará a cabo, hasta mover un método de una clase a otra para así conseguir un acceso mas optimizado y una estructura mas lógica.

Hay una serie de refactorizaciones sencillas pero que son de las que mas se realizan, debido a que solucionan los code smells que más se cometen:

\begin{itemize}
\item EXTRACT METHOD: Es el proceso por el cual, tras detectar que ciertas porciones de código se repiten en distintos métodos o incluso dentro de un mismo método en partes distintas del mismo, pero que tienen la misma estructura y en definitiva cumplen la mismo función, por lo que esas porciones de código repetido pueden ser extraídas para crear un método nuevo, de tal manera que cuando se necesite hacer el proceso que exija usar ese mismo proceso, bastará con llamar al método. 

Así se consigue una estructura más robusta en el código, así como mejor mantenibilidad del mismo, ya que si en un momento dado es necesario modificar esas lineas, en vez de modificarlas en cada una de las localizaciones donde estuvieran, solo se hará en el método nuevo.

Un ejemplo gráfico donde se ve lo que se acaba de explicar:

\imagen{extractM}{Ejemplo sobre la técnica de refactorización \textit{extract method} \cite{extractM}}



\item MOVE METHOD: En este caso, durante la detección de los posibles problemas que pueda tener el código, se observa que un método que se encuentra en una clase, no esta correctamente implementado en la misma, ya que a lo mejor esta haciendo mas uso de recursos de otra clase distinta, y así se realiza, se decide mover ese método a la clase en la que debería estar implementado, ya que esto hace que la estructura sea más lógica, y por consiguiente mas óptima.

Un ejemplo gráfico donde se ve lo que se acaba de explicar:

\imagen{moveM}{Ejemplo sobre la técnica de refactorización \textit{move method} \cite{moveM}}

\item RENAME: Este tipo de refactorización es el más sencillo y a la vez es el que más necesario resulta en muchas ocasiones, ya que se basa en renombrar una variable, método o clase para que tenga más sentido con respecto a la labor que desempeñe. El hecho de que existan nombres excesivamente genéricos o que directamente no cumplan con lo que es realmente es un error muy frecuente, por tanto, esta refactorización es muy frecuente.
\end{itemize}

\subsubsection{Migración}

El concepto de migración puede ser muy extenso, pero en este caso trataremos acerca de la migración en el software.

Este tipo de migración se realiza cuando se quiere cambiar por completo un software para pasar a utilizar otro que este mas actualizado, o cumpla mejor con las necesidades que exija el proyecto en cuestión. 

Para relacionar la técnica con este proyecto, hay que mencionar Android, ya que esta técnica se ha aplicado para realizar la migración desde Android hacia AndroidX, ya que éste último es la versión mejorada del anterior. Pero no es una simple actualización, ya que se encuentran en librerías diferentes, pero esto se explicará más adelante en el apartado de herramientas.


\subsection{Herramientas}

Como ya se ha comentado antes sobre las técnicas y herramientas utilizadas, solo se destacará las que han tenido una verdadera relevancia para este proyecto, ya que hay algunas que ya se mencionan en las versión anteriores \cite{tfg1} \cite{tfm1} de TourPlanner.

\subsubsection{Android}

En esta sección de Android, realmente se verán dos herramientas, que son Android como sistema operativo \cite{android1} y Android Studio como herramienta de desarrollo \cite{as1}. Igualmente, 

\textbf{Android:} sistema operativo que ha sido desarrollado por google, basado en el kernel de linux. Está diseñado inicialmente para dispositivos móviles, aunque ya han alcanzado desarrollos para tabletas, relojes inteligentes e incluso televisiones \cite{android2}.

\textbf{Android Studio:} entorno de desarrollo integrado oficial para la plataforma Android, que fue anunciado en 2013 y tuvo su primera versión disponible en 2014 \cite{as2}. Algunas caracteristicas de las que aporta Android Studio son:

\begin{itemize}
\item Integración de ProGuard y funciones de firma de aplicaciones.
\item Más especificación a la hora de Programar.
\item Renderizado en tiempo real.
\item Consola de desarrollador: consejos de optimización, ayuda para la traducción, estadísticas de uso.
\item Soporte para construcción basada en Gradle.
\item Refactorización específica de Android y arreglos rápidos.
\end{itemize}

\subsubsection{TexMaker}

Esta es la herramienta que se ha decidido utilizar para el desarrollo de la memoria del proyecto \cite{tex}. 

Nos ofrece la posibilidad de utilizar el lenguaje Latex en un entorno sencillo y rápido.

\subsubsection{Github Desktop}

Herramienta utilizada para tener una gestión agil del desarrollo del proyecto. \cite{git}

Con esta herramienta se consigue realizar commits y llevar un control de versiones sobre el proyecto, pudiendo vincular los directorios sobre los que se trabaje en la aplicación, de forma que cada cambio que se realice se verá reflejado en el programa. Después simplemente se podrán realizar commits de una forma rápida.

\subsubsection{SonarQube}

Es una plataforma que nos permite evaluar el código. Es software libre y usa diversas herramientas de análisis estático de código fuente para obtener métricas que pueden ayudar a mejorar la calidad del código de un programa. \cite{sonar2}

Sus funciones cubren entre otras: Informa sobre código duplicado, estándares de codificación, pruebas unitarias, cobertura de código, complejidad ciclomática, errores potenciales, comentarios y diseño del software.

\subsubsection{Logmein Hamachi}

Esta herramienta es el software que se necesita para poder realizar la conexión entre el ciente y el servidor, ya que simula una conexión directa a través de Internet entre ambos extremos \cite{hamachi}. 



