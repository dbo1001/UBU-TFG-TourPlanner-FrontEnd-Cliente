\capitulo{6}{Trabajos relacionados}

En este apartado se definirán algunos trabajos en los que se ha prestado atención durante el proyecto, bien por parecidos en la interfaz o en la forma de trabajar con los datos calculando rutas de otros tipos.

En la mayoría de casos se aportarán elementos visuales para comprender de una manera rápida las razones de la relación entre proyectos.


\section{Google Maps}

La aplicación, bastante conocida por la capacidad de planificar rutas para realizar trayectos de un origen a un destino. \cite{maps}

En esta aplicación también se pueden ver imágenes de los puntos de interés, compartir opiniones con los usuarios, e incluso observar los horarios de los que disponen ciertos establecimientos. 

Es verdad que no tiene la misma capacidad para el cálculo de rutas, en función a un tiempo limitado o a los gustos del usuario, pero en cuanto a interfaz y funcionamiento tiene suficientes parecidos como para ser considerada.

\imagen{maps}{Imágenes de la interfaz de google maps}

\section{Wikiloc}

Wikiloc es una herramienta que permite descubrir y compartir rutas al aire libre, ya sea a pie, en bici, escalada, en moto, en coche y muchas más. \cite{wikiloc}

En este caso si que se comparte la capacidad de realizar rutas, aunque en este caso más que la capacidad de calcular rutas de manera automática, se centra realmente en la capacidad para guardar las rutas que se realicen y compartirlas.

\imagen{wikiloc2}{Imágenes sobre la interfaz de Wikiloc}

\section{Waze}

Esta aplicación está muy centrada en ayuda para la conducción de vehículos, a través de mapas donde se pueden localizar radares, o donde los usuarios aportan información sobre ciertos sitios en caso de haber sobrecarga de tráfico o para hacer rutas más eficientes. \cite{waze}

Contiene una interfaz donde también se muestra un mapa, para poder localizar todos estos puntos que mencionábamos y para poder evitar zonas de tráfico.
\imagen{waze}{Imágenes de la interfaz de Waze}