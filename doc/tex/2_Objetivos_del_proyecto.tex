\capitulo{2}{Objetivos del proyecto}


Objetivos de mejora:
\begin{itemize}
\item Actualización de la interfaz de proyecto desde un sistema Android obsoleto hacia la version mas actualizada posible del mismo, pudiendo implementar mejoras tanto visuales como funcionales
\item Actualizacion sobre la utilizacion de una galeria de imagenes relacionadas con los puntos de interes, ya que la ya implementada era utilizando el servicio de Panoramio, y habria que valorar si esos servicios siguen estando disponibles o si es necesario valerse de otros servicios, como Flickr o Instagram.
\item Diferenciación entre una ruta y otra si se ha realizado en la misma ciudad, para que la aplicacion calcule lugares diferentes basandose en los ya visitados.
\item Implementacion de aplicaciones externas como Twitter o TripAdvisor para poder descubrir las opiniones o valoraciones de otros usuarios.
\item Diferenciacion entre puntos de interes basandose en la actualizacion del algoritmo que pasa a utilizar horarios de apertura y cierre de los mismos, para que la aplicacion no visualice ningun lugar que no pueda ser visitado por estos factores.
\end{itemize}

Objetivos personales:
\begin{itemize}
\item Aprender a utilizar el entorno y lenguaje de programacion de Android, ya que es un sistema que puede resultar muy util para la vida laboral de un programador.
\item Mejorar en el proceso de la gestion de tareas, con aplicaciones como GitHub.
\item Aprender a utilizar el entorno de desarrollo de documentación LaTeX, ayundandome de la herramienta Texmaker.
\item Aprender a trabajar en modo SCRUM, utilizando Issues y Sprints para la realizacion del proyecto.
\item Aprender a realizar trabajo en equipo, manteniendo comunciacion con el compañero encargado de desarrollar el algoritmo del proyecto, para conseguir experiencia sobre un proyecto mas realista que si se realizara solo.
\end{itemize}

