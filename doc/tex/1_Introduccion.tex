\capitulo{1}{Introducción}

La aplicación sobre la que se desarrollará este proyecto de fin de grado comenzó con el proyecto \textit{Generación de rutas Turísticas personalizadas} \cite{tfg1} y un año después fue ampliado en el trabajo \textit{Ampliación sobre la aplicación para la generación de rutas turísticas personalizadas} \cite{tfm1}. En este caso se trabajará sobre la última versión de la que se dispone, donde se realizarán todas las mejoras y actualizaciones necesarias, ya que se trata de un proyecto realizado sobre tecnologías con varios años de antigüedad.

Una parte del proyecto se basa en la generación de rutas turísticas óptimas respecto a los gustos del usuario, añadiendo el factor de horarios de apertura y cierre de los puntos de interés. Esto se consigue con el desarrollo de una serie de algoritmos que se han desarrollado en el trabajo de BackEnd \cite{tfg2}.

Por el otro lado, la interfaz completa de la aplicación, donde el usuario podrá realizar toda la interacción con el sistema sería la correspondiente a la aplicación de Android. Se ha desarrollado de forma que cualquier usuario que tenga costumbre en el manejo de cualquier dispositivo Android le resulte fácil y accesible.

Para conseguir esto mencionado, ha sido necesario modificar varias librerías que se encontraban obsoletas, tratando de utilizar otras más estándar y así obtener una interfaz más accesible. Un ejemplo de esto es el menú deslizante, que ha sido modificado por completo para obtener uno muy similar a los que nos podemos encontrar en aplicaciones como Google Maps o Gmail.

También se ha tratado de centrar el trabajo hacia la optimización en el uso de recursos y métodos dentro de la aplicación. Esto se verá más desarrollado en el anexo 4.

Para que el proyecto adquiera sentido al completo , ha sido necesario cooperar entre las dos partes durante todo el desarrollo del mismo. Se trata de una aplicación que envía peticiones y datos a un servidor, donde se procesa la información y se realizan los cálculos necesarios, por lo que la conexión entre ambos es importante.

